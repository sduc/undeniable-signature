\documentclass[12pt,a4paper]{article}
%\usepackage[margin=1.2in]{geometry}
\title{MOVA signature project: University Competition Application v2}
\author{Sebastien Duc}
\date{}

\begin{document}
    \maketitle

    \section{Overview}
    We consider an application where several universities are participating to a tournament.
    We can imagine that for the first round of the tournament, universities are organized into
    pairs. Let's consider one pair, and let's denote the two universities by $U_1$ and $U_2$. $U_1$ and $U_2$  are in competition.
    In this application, $U_1$ provides quizzes/challenges/riddles to students in $U_2$. Similarily $U_2$ provides quizzes to students in $U_1$.
    Students in a university are forming teams. We assume that teams have a certain amount of time (for example one week) to complete the current quiz/challenge/riddle. 
    Each team in a university have a different quiz\footnote{If several teams in a university have the same quiz, then they can copy on each other to increase the final score.}.
    When completed they send it to the server of both universities and get a score. Then the university who provided the quiz also provides the solution of the quiz and the score the 
    team did, to the students. At the end of the first round, all the best of the two universities 
    (i.e. the ones with the highest cumulated score of all teams in the university) in the pairs are 
    promoted to the second round. The game goes on the same way until round $n$ with only on pair (only two universities) and the best of the two wins the game.

    \section{Security}
    For this part let us consider one pair with universities $U_1$ and $U_2$ as before. Let us denote the server of $U_1$ by $S_1$ and the server of $U_2$ by $S_2$.
    We can assume that communication between $S_1$ and $S_2$ are confidential, autenticated and preserve integrity.
    When $U_1$ creates a quiz/challenge for students of $U_2$ they sign it with MOVA and then send it to them. In that way, the students are ensured that the quiz comes indeed
    from $U_1$ and that it is not a fake quiz from other cheating students who wants to make them lose time, hence lose score. Furthermore, only the designated team can verify the signature.
    Then, when the students are sending back the quiz to $S_2$, $S_2$ can sign the filled quiz with MOVA and send it to back to the team. The team can verify that the signature 
    is correct\footnote{In that model, we consider that the students don't trust the server and want to be sure that answer were not changed or deleted.},
    and send it to $S_1$. Finally $S_1$ receives the filled quiz and verify both signatures.
    For signatures coming from $S_2$, $S_1$ can run a batch verification with $S_2$.
    Then when $S_1$ provides the solutions with score the the team, he signs everything and send. Therefore, students are ensured that the solution and the score are correct.

    \section{Problems}
    As opposed to the previous version of the application, we don't need to keep authentication and integrity between the students and their university server, because the server send 
    the signed filled quiz back to the team and the team can verify the signature.    
    In the previous version we had to assume that $S_2$ did not modify the results of the quizzes before signing them. This is not an issue anymore, because the students receive 
    their signed filled quiz back.

    \section{Using classical Signatures}
    In this application we could try to use classical signature. For the signature of the unfilled quiz (the signature signed by $S_1$, when $S_1$ is the quiz provider), 
    MOVA could bring something compared to classical signatures, because
    \begin{itemize}
        \item We use that MOVA is not universally verifiable, because only the designated team must verify the signature.
        % \item We don't need the signature to be invisible
        \item But still, we cannot use batch verification
        \item But still, it is not really useful for the signatures to be very short
    \end{itemize}
    For the signature of the filled quiz (signed by $S_2$, when $S_1$ is the quiz provider), we exploit the fact that $S_1$
    runs batch verification. 

    \section{Conclusion}
    To conclude, it seems that MOVA could bring something for this application, but only in a weird threat model.


\end{document}
