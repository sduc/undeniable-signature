\documentclass{beamer}

\usepackage[ngerman,english]{babel}
\usepackage{graphicx,amssymb,latexsym,epsfig,fancyhdr,color,bbding}
\usepackage{verbatim} 
\usepackage{lastpage}
\usepackage{graphicx}
\usepackage{multirow}
\usepackage{epstopdf}
\cfoot{\thepage\ of \pageref{LastPage}}



\newtheorem{proposition}{Proposition}

  \newcommand{\la}{\leftarrow}
  \newcommand{\ra}{\rightarrow}
  \newcommand{\lora}{\Longrightarrow}
  \newcommand{\La}{\Leftarrow}
  \newcommand{\Ra}{\Rightarrow}
  \newcommand{\lla}{\longleftarrow}
  \newcommand{\lra}{\leftrightarrow}
  \newcommand{\Lra}{\Leftrightarrow}
  \newcommand{\hra}{\hookrightarrow}
  \newcommand{\da}{\!\!\downarrow}
  \newcommand{\ua}{\!\!\uparrow}
  \newcommand{\uhr}{\upharpoonright}
  \newcommand{\ov}{\overline}
  \newcommand{\un}{\underline}
  \newcommand{\rvec}[1]{\stackrel{\ra}{#1}}
  \newcommand{\lvec}[1]{\stackrel{\la}{#1}}
  \newcommand{\lag}{\langle}
  \newcommand{\rag}{\rangle}
  
  \def\cA{{\mathcal A}}
  \def\cB{{\mathcal B}}
  \def\cC{{\mathcal C}}
  \def\cD{{\mathcal D}}
  \def\cE{{\mathcal E}}
  \def\cF{{\mathcal F}}
  \def\cG{{\mathcal G}}
  \def\cH{{\mathcal H}}
  \def\cI{{\mathcal I}}
  \def\cK{{\mathcal K}}
  \def\cL{{\mathcal L}}
  \def\cM{{\mathcal M}}
  \def\cN{{\mathcal N}}
  \def\cO{{\mathcal O}}
  \def\cP{{\mathcal P}}
  \def\cQ{{\mathcal Q}}
  \def\cR{{\mathcal R}}
  \def\cS{{\mathcal S}}
  \def\cT{{\mathcal T}}
  \def\cV{{\mathcal V}}
  \def\cU{{\mathcal U}}

  \def\bA{{\bf A}}
  \def\bB{{\bf B}}
  \def\bC{{\bf C}}
  \def\bD{{\bf D}}
  \def\bF{{\bf F}}
  \def\bG{{\bf G}}
  \def\bH{{\bf H}}
  \def\bI{{\bf I}}
  \def\bP{{\bf P}}
  \def\bN{{\bf N}}
  \def\bQ{{\bf Q}}
  \def\bR{{\bf R}}
  \def\bS{{\bf S}}
  \def\bT{{\bf T}}
  \def\bU{{\bf U}}
  \def\bV{{\bf V}}
  \def\bZ{{\bf Z}}




\useoutertheme{infolines}
\setbeamertemplate{footline}{\scriptsize{\hfill\insertframenumber\hspace*{.5cm}} } 


\begin{document}



\begin{frame}

\begin{center}
    \includegraphics[width=2cm]{logo_epfl_coul.eps}
  \end{center}
  \vspace{0.3cm}
  
   \begin{center}
  {\LARGE  Primeless Cryptography}
  \end{center}
  \vspace{0.1cm}


 \begin{center}
   Sonia Mihaela Bogos
  \end{center}
  \vspace{0.1cm}
 

 \begin{center}
   School of Computer and Communication Sciences 
  \end{center}
  \vspace{0.1cm}

  \begin{center}
    \begin{tabular}{cc}
      \begin{tabular}{p{4.0cm}}
         \begin{center}
    \begin{normalsize}{
        \bf Responsible}\\Prof. Serge Vaudenay\\EPFL/LASEC
    \end{normalsize}
  \end{center}
      \end{tabular}&
      \begin{tabular}{p{4.0cm}}
          \begin{center}
    \begin{normalsize}{
        \bf Coordinator}\\Ioana Boureanu\\EPFL/LASEC
    \end{normalsize}
  \end{center}
      \end{tabular}
    \end{tabular}
  \end{center}
  
  \begin{center}
    \includegraphics[width=2.5cm]{logo_lasec_coul.eps}
  \end{center}
 
\end{frame}

%%%
%%%
%%%

\begin{frame}
\frametitle{Outline}

 	\begin{itemize}
		\item {\bf Motivation} 
		\item Background
		\item A first attempt
		\item New Proposal
		\item Conclusion
	\end{itemize}
\end{frame}


%%%
%%%
%%%

\begin{frame}
\frametitle{Motivation (1)}
    The goal of this report is to analyse and construct cryptographic primitives that use
    primeless cryptography. 
    
    \bigskip
    
    What is \textbf{primeless cryptography}?
    
    The universe of cryptography where prime numbers are not generated. In our case, 
    random numbers are used.
    	


\end{frame}

%%%
%%%
%%%

\begin{frame}
\frametitle{Motivation (2)}

	The purpose is to reduce the asymptotic complexity. For example, 
	RSA has a complexity of:
	
	\begin{itemize}
		\item Setup: $O(l^4)$
		\item Encryption: $O(l^3)$
		\item Decryption: $O(l^3)$	
	\end{itemize}
	when we generate two different prime numbers, $p$ and $q$, 
	of size $l$.
	
	
\end{frame}

%%%
%%%
%%%

\begin{frame}
\frametitle{Outline}

 	\begin{itemize}
		\item Motivation 
		\item \textbf{Background}
		\item A first attempt
		\item New Proposal
		\item Conclusion
	\end{itemize}
\end{frame}


%%%
%%%
%%%

\begin{frame}
\frametitle{Characters}


 	\begin{definition}
 	Let $G$ be an Abelian group. A \textbf{character} $\chi$ is a function 
 	$\chi : G \rightarrow \mathbb{C} \backslash \{ 0 \}$ such that \\
 	
 	\hspace{30mm}  $\chi(a+b)= \chi(a) \cdot \chi(b)$, $\forall a,b \in G$.
 	\end{definition} 
 	
 	\bigskip
 	
 	\begin{itemize}
 		\item group structure over the set of all characters on $G$. 
 		\item we use characters of order 2 and 4.
 		\item denote by $\varepsilon$ the trivial character
 				where $\varepsilon(x)=1$, $\forall x \in G$.
 	\end{itemize}
 	

\end{frame}

% % %
% % %
% % %

\begin{frame}
\frametitle{Characters of order 2}

	\begin{itemize}
		\item $p \in \mathbb{Z}$ prime.
		
		\item two characters in 
			$Z_p^*$, solutions of $\chi^2= \varepsilon$: Legendre
			symbol, $(\frac{.}{p})$, and $\varepsilon.$
		\item Jacobi - extension of the Legendre symbol.
		\item a list of four characters, solutions of $\chi^2= \varepsilon$, in $\mathbb{Z}_n^*:$ $\varepsilon$, $(\frac{.}{n})$, $(\frac{.}{p})$ and $(\frac{.}{q})$, 
		for $n= p\cdot q$ with $p$ and $q$ two different odd primes.
	\end{itemize}

	
	
\end{frame}

% % %
% % %
% % %

\begin{frame}
\frametitle{Gaussian Integers}

	\begin{itemize}
		\item $\mathbb{Z}[i]= \{ a + bi$  $|$  $a,b \in \mathbb{Z}, i^2=-1 \}$.
		\item for $\alpha = a+bi$, $N(\alpha) = \alpha \cdot \bar{\alpha} = a^2 + b^2$.
	\end{itemize}
	
	
	\begin{proposition}
		Value $\alpha$ is a prime in  $\mathbb{Z}[i]$, iff 
		$\alpha$ satisfies one of the following:
		\begin{itemize}
			\item $\alpha = 1+i$ or $\alpha= 1-i$.
			\item $\alpha$ is prime in $\mathbb{Z}$ and $\alpha \equiv 3 \pmod{4}$.
			\item $\alpha \cdot \bar{\alpha}$ is a prime in $\mathbb{Z}$ and 
			$\alpha \cdot \bar{\alpha} \equiv 1 \pmod{4}$.
		\end{itemize}  
	\end{proposition}
	


\end{frame}

% % %
% % %
% % %

\begin{frame}
\frametitle{Quartic Residue Symbol (1)}

\begin{definition}
 				Let $\alpha,\beta \in \mathbb{Z}[{i}]$ be such that $(1+i) \nmid \beta$. Values $\beta$ and $\alpha$ are relatively prime. 
 				 
 				\bigskip
 				 				 
 				$\mathbf{\chi}_{\beta} : \mathbb{Z}[{i}] \to \{\pm 1, \pm i\}$: 
 				\begin{itemize}
 					\item $\mathbf{\chi}_{\beta} = (\alpha^{\frac{N(\beta)-1}{4}}\big)\bmod \beta,$ for $\beta$ prime.
 					\item $\mathbf{\chi}_{\beta} = \prod _{i} \mathbf{\chi}_{\beta _{i}}(\alpha),$ for $\beta = \prod _ {i} \beta _{i},$ where $\beta_{i}$ is prime.
 				\end{itemize}
\end{definition}

\end{frame}

% % %
% % %
% % %

\begin{frame}
\frametitle{Quartic Residue Symbol (2)}


 	\begin{definition}
 		A nonunit $\alpha \in \mathbb{Z}[i]$ is primary if $\alpha \equiv 1 \pmod{(1+i)^3}$.
 	\end{definition}
 	
 	\begin{proposition}
 	 				Let $\alpha,\beta \in \mathbb{Z}[{i}]$ be such that $(1+i) \nmid \beta$, $\gcd(\beta,\alpha) = 1$ and $\gcd(\beta,\alpha') = 1$.
 	 				\begin{itemize}
 	 					
 	 					\item (Multiplicativity) $\chi_{\beta}(\alpha \alpha') = \chi_{\beta}(\alpha) \cdot \chi_{\beta}(\alpha')$.
 	 					\item (Modularity) If $\alpha \equiv \alpha' \pmod{\beta}$, $\chi_{\beta}(\alpha) = \chi_{\beta}(\alpha')$.
 	 					\item (Quartic Reciprocity Law) If $\alpha, \beta$ are primary, 
 	 						
 	 						\hspace{30mm} $\chi_{\beta}(\alpha) = \chi_{\alpha}(\beta) \cdot (-1)^{\frac{N(\alpha)-1}{4} \cdot \frac{N(\beta)-1}{4}}$.
 	 					\item (Complementary Laws) If $\beta = a + bi$ is primary,
 	 					
 	 						\hspace{30mm} $\chi_{\beta}(i)= i^{\frac{N(\beta)-1}{2}}$,  $\chi_{\beta}(1+i) = i^{\frac{a-b-b^2-1}{4}}$.
 	 						
 	 				\end{itemize}
 	 				
 	 			\end{proposition}
 	 			
 	 			
 		
\end{frame}

% % %
% % %
% % %



\begin{frame}
\frametitle{Algorithm to compute the quartic residue symbol \cite{oswald}}

	$t \gets 0$\\
	\textbf{While} $(N(\alpha) > 1)$ \textbf{do}
	\begin{itemize}
	\item[] $\alpha \gets \alpha \bmod \beta~$~~~~~~~~~~~~~~~~~~~~~~~~~~~~~~~~~~~~(Modularity)
	\item[] find $\alpha'$ primary such that $\alpha= i^j \cdot (1+i)^k \cdot \alpha'$
	\item[] set $\alpha \gets \alpha'$ and adjust $t$~~~~~~~~~~~~~~~~~~~~~~~~~(Multiplicativity)
	\item[] swap $\alpha$ and $\beta$ and adjust $t$~~~~~~~~~~~~~~~~~~~~~(Reciprocity Law)
	\end{itemize} 
	\textbf{If} $(N(\alpha) = 1)$ return $i^t$

	%there are characters of other orders
\end{frame}

% % %
% % %
% % %

\begin{frame}
\frametitle{Outline}

 	\begin{itemize}
		\item Motivation 
		\item Background
		\item \textbf{A first attempt}
		\item New Proposal
		\item Conclusion
	\end{itemize}
\end{frame}

% % %
% % %
% % %

\begin{frame}
\frametitle{Scheme 1 \cite{p}}
 	

 	\textit{Key generation} \\

 	\textbf{Input:} Security parameter $s$. \\
 		 \textbf{Output:} Public key: $(n, p)$;  Private key: $\gamma$.
 		 
 		\bigskip
 		 
 			\quad 1. Select a big $\gamma \in \mathbb{Z}[i],$ $i.e.$ $\gamma = a' + b'i$, where $a', b' \in \mathbb{Z}$ and
 		the size of $a', b'$ depends on the security parameter $s$ . \\
 		 \quad 2. Compute $n =\gamma \bar{\gamma}$ ($n \in \mathbb{Z}$). \\
 		 \quad 3. Pick a $p \in \mathbb{Z}$ such that  $ \chi_{\gamma} (p) = i $. \\



\end{frame}

% % %
% % %
% % %

\begin{frame}
\frametitle{Scheme 1 \cite{p}}
	
		\textit{Encryption} \\

		 \textbf{Input:} a bit $b$. \\
			 \textbf{Output:} the encryption $c$, $c  \in \mathbb{Z}_n$.  \\
			 \textbf{Public key:} $(n,p)$.
			 
			 
		\bigskip
			 
		\quad  1. Pick an $a \in \mathbb{Z}$ such that \textbf{if} $b =0$ \textbf{then} $a \equiv 1 \pmod{4}$
		\textbf{else}  $a \equiv 3 \pmod{4}$. \\
	    \quad 2. $c \equiv p^a \pmod{n}$. \\ 
		
		\bigskip
		
		
		\textit{Decryption} \\
		
		 \textbf{Input:} the encryption $c$, $c \in \mathbb{Z}_n$. \\
				 \textbf{Output:} a bit $b$. \\
				 \textbf{Secret key:} $\gamma$.
				 
		\bigskip
			
			
		 \quad  1. Compute the quartic residue symbol $ \chi_{\gamma} (c)$. \\   
		\quad   2. \textbf{if} $\chi_{\gamma} (c) =i$ \textbf{then} $b = 0$  \textbf{else} {$b = 1$\;}.\\

\end{frame}

% % %
% % %
% % %

\begin{frame}
\frametitle{Weakness of Scheme 1}

	\begin{itemize}
		\item  an adversary 
		  $\mathcal{A}$ does not need to know $\gamma$, 
		  but only an \textit{equivalent} key $\gamma'$
		  to $\gamma$. 
		\item  equivalent key: a gaussian factor $\gamma' \in \mathbb{Z}[i]$ of $n$ with
		  $\chi_{\gamma'}(p) = \pm i$.
		\item for a prime $f \equiv 1 \pmod{4}$, $f = \gamma' \cdot \bar{\gamma'} $, with probability of $\frac{1}{2}$,
		  $\chi_{\gamma'}(p) = \pm i$.    
		\item  $Prob_{n \in \mathbb{Z}, 2 \not | n}[ f_1 \equiv 1 \pmod{4}, f_2 \equiv 1 \pmod{4} , f_1 \not= f_2 $   $ ] = \frac{1}{ 2^2 \cdot f_1 \cdot f_2}$.    
	\end{itemize}
 
  \bigskip
  
 
  		
  
  


\end{frame}

% % %
% % %
% % %


\begin{frame}
\frametitle{Breaking Scheme 1}
 
     \textbf{Input:} the encrypted bit $c$, $c \in \mathbb{Z}_n$. \\
 		    			 \textbf{Output:} a bit $b$.\\
 		    			 \textbf{Public key:} $(n,p)$.
 		    	\bigskip
 		    	
 		    	
 		    	\quad  1. Find two small prime factors of $n$, $f_j \in \mathbb{Z}$, such that $f_j \equiv 1 \pmod{4}$ for $1 \leq j \leq 2$.  \\
 		        \quad 2. Run \textit{Cornacchia} and \textit{Tonelli} algorithms to find  \\
 		        \quad \quad $\gamma_j \in \mathbb{Z}[i]$ such that $\gamma_j \cdot \bar{\gamma_j} = f_j$. \\
 		         \quad 3. \textbf{if} $\chi_{\gamma_j} (p) =i$ or $\chi_{\gamma_j} (p) = -i$ \textbf{then} decrypt the value $c$ using the following rules:  \\
 		         \quad \quad for $\chi_{\gamma_j} (p) =i$:  \textbf{if} $\chi_{\gamma_j} (c) =i$ \textbf{then} $b=0$ \textbf{else} $b=1$ \\
 		         \quad \quad for $\chi_{\gamma_j} (p) = -i$:  \textbf{if} $\chi_{\gamma_j} (c) = -i$ \textbf{then} $b=0$ \textbf{else} $b=1$ \\ 
 		         \quad \quad \textbf{otherwise} abort. 
  
\end{frame}

% % %
% % %
% % %

\begin{frame}
\frametitle{Outline}

 	\begin{itemize}
		\item Motivation 
		\item Background
		\item A first attempt
		\item \textbf{New Proposal}
		\item Conclusion
	\end{itemize}
\end{frame}

% % %
% % %
% % %

\begin{frame}
\frametitle{Key Agreement}

  	\small{	
  		\begin{equation*}
  		A \qquad  \qquad \qquad \qquad \qquad \qquad \qquad B
  		\end{equation*}
  		
  		\noindent pick $\alpha = \prod_{i=1}^{i=k} \alpha_i,$ where $\alpha_i \in \mathbb{Z}$ \\
  		pick $\beta = \prod_{i=1}^{i=k} \beta_i,$ where $\beta_i \in \mathbb{Z}$ \\
  		compute $n = \alpha \cdot \beta $
  				
  		\bigskip
  				
  		\noindent 
  		pick $x_1, x_2, \ldots x_t \in \mathbb{Z}_n^*$ \\
  		compute $y_i = (\frac{x_i}{\alpha}), 1 \leq i \leq t$
  				 
  		\begin{equation*}
  		\qquad \xrightarrow[y_1,y_2, \ldots, y_t]{\hspace*{1.8cm} n,x_1,x_2,\ldots,x_t \qquad \qquad}  
  		\end{equation*}
  				
   
  				
  		\hspace{80mm} pick $b_1, b_2, \ldots, b_t \in \{ 0,1\}$ 
  				
  		\hspace{80mm} $z= x_1^{b_1} \cdots x_t^{b_t} \pmod{n}$
  				
  				
  		\begin{equation*}
  		\qquad \xleftarrow{\hspace*{1.8cm} z \qquad \qquad}  
  		\end{equation*}   
  				
  		$K = (\frac{z}{\alpha})$ \hspace{80mm} $K = y_1^{b_1} \cdots y_t^{b_t}$
  		
  		  \begin{center}
  		\textbf{Scheme 2}
  		\end{center}		
  	}	

\end{frame}

% % %
% % %
% % %

\begin{frame}
\frametitle{Choice of parameters}

	\begin{itemize}
		\item $s \in \mathbb{Z}$ security parameter.
		\item $\alpha_i$ and $\beta_i$ are of size $l$, for $1 \leq i \leq k$. (See next slides)
		\item value $t$ assures the uniqueness of character $(\frac{.}{\alpha})$ and $t\geq s$. 
		\item $x_i's$ randomly chosen from $\mathbb{Z}_n^*$, $y_i's$ not all equal to 1.
		\item $b_i's$ not all equal to 0. 
	\end{itemize}

	\bigskip

 Complexity: $O(tk^2l^2)$.

\end{frame}

% % %
% % %
% % %

\begin{frame}
\frametitle{Security (1)}

	Scheme 2 is secure if, after the run of it,
			
	\hspace{10mm} $Adv_{\mathcal{A}}(s) = Prob [K = K'] - \frac{1}{2}$,  \\
	
	\bigskip
	
	is negligible in terms of $s$, where $K'$ is the bit guessed by $\mathcal{A}$ and $K$ is the agreed key. 
\end{frame}

% % %
% % % 
% % %

\begin{frame}
\frametitle{Computational Problems}

	\textbf{Factorization.} Given $n \in \mathbb{Z}$, find the prime factorization of $n$.
 	
 	\bigskip	
 			
 	\textbf{MOVA$^d$}. Let $n \in \mathbb{Z}$, $t$ be a positive integer and $\chi$ be a \textit{hard character} of order $d$ on $\mathbb{Z}_n^*$. Given $t$ pairs of the form $(x_i, \chi(x_i))$, with $1 \leq i \leq t$, and given $x \in \mathbb{Z}_n^*$, compute $\chi(x)$. 
 	
 	\bigskip
 	
 	\begin{proposition}
 		We have the following Karp reduction:
 		 	
 		 	\hspace{30mm} 	MOVA$^d$ $\leq$ Factorization.	
 	\end{proposition}
 	 		
  	%as a motivation to have the factorization hard	 
\end{frame}

% % %
% % %
% % %

\begin{frame}
\frametitle{Factorization for random numbers (1)}

\begin{table}
 		\begin{center}
 		
 		\begin{tabular}{ | c | c | c |}
 		
 		\hline
 		Prob. $n_i < n^x$  & $x$ for $n_1$ & $x$ for $n_2$ \\
 		\hline
 		0.01 & $0.26974$ & $0.00558$ \\
 		\hline
 		0.02 &  $0.29341$& $0.0111$ \\
 		\hline
 		0.10 & $0.37851$ & $0.05308$ \\
 		\hline 
 		0.50 & $0.606$ & $0.21172$ \\
 		\hline
 		0.90 & $0.90484$ & $0.35899$ \\
 		\hline
 		
 		\end{tabular}
 		\label{res}
 		\caption{Distribution size for the largest two prime factors \cite{fact}}  
 		\end{center}
 		\end{table}
 		
 		It is necessary to have the following inequality: 
 			
 		 \begin{equation} 
 			 			   %\label{eqsecn}
 			 				 \frac{C_{ECM}(x \cdot l)}{F_2(x)} \geq C_{GNFS}(L') \cite{thesis},
 			 			   \end{equation}
 			 			   
 			 		where $l$ is the size of $n$,  $0<x<1$ and  $F_2(x)=$ Prob  $[n_2 < n^x]$.	   
 			
 		  
 			   
\end{frame}

% % %
% % %
% % %

\begin{frame}
\frametitle{Factorization for random numbers (2)}

	\begin{itemize}
		\item take $n$ to be a product of random numbers $p_i$:
		
		\hspace{10mm} $n= p_1 p_2 \cdots p_k$, with $|p_i| =l$ for all $1 \leq i \leq k$, $|n| = k \cdot l$. 
		
		\item modification of $(1)$: 
		
		\hspace{30mm} $min_{0 < x < 1}(C_{ECM}(x \cdot l), F_2(x)^{-k}) \geq C_{GNFS}(L')$.
		
		\item sufficient condition:
		
		$min [F_2(x)^{-k}, min_{x \leq u \leq y} \frac{C_{ECM}(ul)}{F_2(u)^k}, C_{ECM}(yl)]$ $\geq C_{GNFS}(L').$ \quad  $(2)$ 
		
	\end{itemize}	
	
	
	
	\bigskip
	
	The inequality $(2)$ can be simplified with:
	
	\hspace{30mm} $C_{ECM}(x \cdot l) = F_2(x)^{-k} = C_{GNFS}(L')$. 
	
			
\end{frame}

% % %
% % %
% % %

\begin{frame}
\frametitle{Factorization for random numbers (3)}

	We obtain the following approximations for the size of $n$:
	
	
		  	\begin{table}
	 		\begin{center}
	 		
	 		\begin{tabular}{ | c | c | c | c | c |}
	 		
	 		\hline
	 		  $x$  & $F_2(x)$ & $l$ & $k$ & $|n| = k \cdot l$ \\
	 		\hline
	 		0.21172 & 0.5 &  1$\,$276 & 45 &  57\,420 \\
	 		\hline
	 		0.01110 & 0.02 & 24\,500& 8  & 196\,000 \\
	 		\hline
	 		0.40681 & 0.96 & 660 & 750 & 495\,000 \\
	 		\hline 
	 		
	 		\end{tabular}
	 		\label{table2}
	 		\caption{Required size of $n$ for $L'= 1024$}  
	 		\end{center}
	 		\end{table}
	 		
	%with the conclusion that we need numbers of hundreds of thousands of bits

\end{frame}


\begin{frame}
\frametitle{Security (2)}

\begin{theorem}
				Let $\varphi: G  \rightarrow \mathbb{Z}_d$ be a group homomorphism. If one can compute a $f$ such that 
				$Prob_{z \in G}(f(z) \not= \varphi(z)) \leq \frac{\xi}{12}$  with a constant $\xi < 1$, then one can compute
				$\varphi$ in a number of calls to $f$ bounded by a polynomial in log$(\# G)$ \cite{mova} . 
\end{theorem}

\bigskip

\begin{corollary}[Security of Scheme 2]
	Assuming that Factorization and MOVA$^d$ are hard, then no adversary $\mathcal{A}$ can guess the bit $K$ from one run of the agreement but with a probability bounded by 
	$\frac{1}{2}$. 	
\end{corollary}
 	
\end{frame}



% % %
% % %
% % %


\begin{frame}
\frametitle{Probabilistic Encryption}

\small{

\begin{equation*}
		A \qquad  \qquad \qquad \qquad \qquad \qquad \qquad B
		\end{equation*}
			pick $\alpha = \prod_{i=1}^{i=k} \alpha_i,$ where $\alpha_i \in \mathbb{Z}$ \\
			pick $\beta = \prod_{i=1}^{i=k} \beta_i,$ where $\beta_i \in \mathbb{Z}$ \\
			compute $n = \alpha \cdot \beta $
			
		\bigskip		
				
		\noindent 
		pick $x_1, x_2, \ldots x_t \in \mathbb{Z}_n^*$ \\
		compute $y_i = (\frac{x_i}{\alpha}), 1 \leq i \leq t$
				 
		\begin{equation*}
		\qquad \xrightarrow[y_1,y_2, \ldots, y_t]{\hspace*{1.8cm} n,x_1,x_2,\ldots,x_t \qquad \qquad}  
		\end{equation*}
				
 
				
		\hspace{80mm} pick $b_1, b_2, \ldots, b_t \in \{ 0,1\}$ 
				
		\hspace{80mm} $K = y_1^{b_1} \cdots y_t^{b_t}$ 			

				
						

		\begin{equation*}
		\qquad \xleftarrow{\hspace*{1.8cm} z= x_1^{b_1} \cdots x_t^{b_t} \pmod{n}  \qquad \qquad}  
		\end{equation*}   
				
		$K = (\frac{z}{\alpha})$ 
		
		\begin{center}
		\textbf{Scheme 3}
		\end{center}
	}	

\end{frame}

% % %
% % %
% % %

\begin{frame}
\frametitle{Using characters of order 4}

	\begin{itemize}
		\item improve the schemes by using characters of order 4: agree on a 2 bit value $K$ (or encrypt two bits).
		\item compute $n$ as $ \prod_i \alpha_i \cdot \bar{\alpha_i} \cdot \beta_i \cdot \bar{\beta_i}$.
		\item $b_i's$ are randomly chosen from $\{ 0,1,2,3\}$.
		\item correctness and the security proof are maintained. 	
	\end{itemize}
\end{frame}

% % %
% % %
% % %

\begin{frame}
\frametitle{Outline}

 	\begin{itemize}
		\item Motivation 
		\item Background
		\item A first attempt
		\item New Proposal
		\item \textbf{Conclusion}
	\end{itemize}
\end{frame}

% % %
% % %
% % %

\begin{frame}
\frametitle{Conclusion}

	\begin{itemize}
		\item two cryptographic primitives, that use primeless cryptography, were introduced.
		\item the need to have the factorization problem hard.
		\item smaller complexity for the setup of $n$.
		\item hard to work with too big numbers.		
	\end{itemize}
	
	\bigskip
	
	Future work:
	\begin{itemize}
		\item asymptotic analysis on values $k$ and $l$.
		\item construction of a more efficient cryptographic primitive.
	\end{itemize}

\end{frame}

% % %
% % %
% % %


\bibliographystyle{plain}
\bibliography{biblio2}


\end{document}




