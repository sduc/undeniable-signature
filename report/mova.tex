\chapter{MOVA Signature Scheme}
\label{chap:mova}

In this chapter, we will first present undeniable schemes. 
Then MOVA signature scheme will be described herein.

\section{Undeniable Signatures}
Undeniable signature scheme were invented by Chaum and van Antwerpen~\cite{cite:chaum}.
The aim of an undeniable signatures scheme is roughly the same that a classical 
signature scheme (see section~\ref{sec:signature}) which is to bind an identity to a message.
The only difference is that the verifier must run an interactive protocol with the signer
to verify the validity of the message-signature pair.

As in classical signatures, let $n$ be a security parameter. Let $S$ and $V$ be
the signer and the verifier respectively. To define an undeniable scheme, the following algorithms must be 
defined:
\begin{description}
    \item[Setup] It consists of two algorithms that generate the key pair for $S$ and $V$.
        Let $(k_p^S,k_s^S) \leftarrow \mathrm{Setup}^S(1^n)$ and 
        $(k_p^V,k_s^V) \leftarrow \mathrm{Setup}^V(1^n)$.
    \item[Sign] An algorithm to sign a message in the message space using the secret key 
        of $S$.
        Let $\sigma \leftarrow \mathrm{Sign}(m,k_s^S)$ ,where $m$ is the message.
    \item[Confirm] It consists of an interactive protocol between $S$ and $V$ to confirm 
        the validity of a message signature pair.
    \item[Deny] It consists of an interactive protocol between $S$ and $V$ to deny the 
        validity of a message signature pair.
\end{description}


\section{MOVA}
The MOVA\footnote{MOVA stands for {\bf Mo}nnerat {\bf Va}udenay} signature scheme 
was invented by J. Monnerat and S. Vaudenay~\cite{cite:thesis-monnerat,cite:2-move,
cite:mova-crypto-journal,cite:opti-mova,cite:generic-mova}. It is an undeniable signature scheme. 

\subsection{Definition}
Consider a signer $S$ and a verifier $V$. Assume that they can use two pseudorandom generators,
$\GenK$ and $\GenS$. 

The {\bf Setup} algorithm is defined as follows.
$S$ chooses two Abelian groups $\Xgroup$ and $\Ygroup$. Then he chooses 
$\hom: \Xgroup \rightarrow \Ygroup$. Then using $\GenK$ on seed $\seedK$ he generates 
$\Lkey$ elements of $\Xgroup$:
\[
\Xkey{1},...,\Xkey{\Lkey} \leftarrow \GenK(\seedK).
\]
Finally, he computes $\Ykey{1},...,\Ykey{\Lkey}$, where $\Ykey{i} = \hom(\Xkey{i})$.

With this Setup algorithm, the public key 
\[
    K_p^{S} = (\Xgroup,\Ygroup,d,\seedK,(\Ykey{1},...,\Ykey{\Lkey})), 
\]
where $ d = |\Ygroup|$ and the secret key
\[
    K_s^{S} = \hom.
\]

The {\bf Sign} algorithm is defined as follows.
Let $m \in \{0,1\}^*$ be the message to sign. First $\GenS$ is seeded with $m$ to obtain
$\Xsig{1},...,\Xsig{\Lsig} \in \Xgroup$. The signature is 
\[
    \sigma = (\Ysig{1},...,\Ysig{\Lsig}),
\]
where $\Ysig{i} = \hom(\Xsig{i})$.

The {\bf Confirm} interactive protocol is defined as follows.
It takes $(m,\sigma)$ as input. First $S$ and $V$ retrieve values $\Xkey{i},\Xsig{j}$ for
$i = 1,...,\Lkey$ and $j = 1,...,\Lsig$ by using $\GenK$ and $\GenS$.
Then they run $\mathrm{GHIproof}_{\Icon}(S)$ with 
\[
S = \{(\Xkey{i},\Ykey{i})| i = 1,...,\Lkey\} \cup \{(\Xsig{i},\Ysig{i})| i = 1,...,\Lsig\}
\]
and $S$ is playing the role of the prover.
$\mathrm{GHIproof}_{l}(S)$ is described in figure~\ref{fig:ghiproof}.

The {\bf Deny} interactive protocol is defined as follows.
It takes a supposedly invalid message-signature pair $(m,\sigma')$ as input with $\sigma' = (\Zsig{1},...,\Zsig{\Lsig})$
First, $S$ and $V$ retrieve values $\Xkey{i},\Xsig{j}$ for
$i = 1,...,\Lkey$ and $j = 1,...,\Lsig$ by using $\GenK$ and $\GenS$.
Then they run  $\mathrm{coGHIproof}_{\Iden}(S,T)$ with 
\[
S = \{(\Xkey{i},\Ykey{i})|i=1,...,\Lkey\}
\]
and 
\[
T = \{(\Xsig{i},\Zsig{i})|i=1,...,\Lsig\}.
\]
$\mathrm{coGHIproof}_{\Iden}(S,T)$ is described in figure~\ref{fig:coghiproof}. 

\subsection{Interactive Proofs}
In this section, we will describe the two interactive proofs used in MOVA (namely GHIproof and
coGHIproof). In MOVA the secret key is a group homomorphism.
So let us start with a small definition.
\begin{definition}
    Let $G$ and $H$ be two Abelian groups. Let us consider the set of points  $S~=~\{(x_i,y_i)|i=1,...,s\}~\subseteq~G\times H
    $. We say that $S$ interpolates in a group homomorphism if there exists $\hom:G\rightarrow H$ st. 
    $\hom(x_i) = y_i$, for $i = 1,...,s$.
    Furthermore if we consider $T \subseteq G \times H$, we say that $T$ interpolates in a group homomorphism with the set $S$ if $S\cup T$ interpolates in a group homomorphism.
\end{definition}
We can now define two problems taken from~\cite{cite:generic-mova,cite:2-move} related to MOVA and the previous definition.
We have the Group Interpolation Problem:
\begin{prob_descr}
    \item[$n$-$S$-GHI Problem]
    \item[Parameters:] Two abelian groups $G,H$, and $S \subseteq G\times H$, $n\in\N$.
    \item[Instance Generation:] $n$ elements $x_1,...,x_n \in_U G$.
    \item[Problem:] Find $y_1,...,y_n \in H$ such that $\{(x_i,y_i)|i=1,...,n\}$ interpolates
        with $S$ in a group homomorphism.
\end{prob_descr}
And we have the Group Interpolation Decisional Problem:
\begin{prob_descr}
    \item[$n$-$S$-GHID Problem]
    \item[Parameters:] Two abelian groups $G,H$ and $S \subseteq G\times H$, $n\in\N$.
    \item[Instance Generation:] The instance $T$ is generated according to $T_0$ or $T_1$.
        $T_0$ is generated by picking $\{(x_i,y_n)|i=1,...,n\} \in (G\times H)^n$ uniformly at
        random such that it interpolates with $S$ in a group homomorphism. $T_1$ is picked
        uniformly at random in $(G\times H)^n$.
    \item[Problem:] Decide whether the instance $T$ is of type $T_0$ or $T_1$.
\end{prob_descr}
For both problems, in practice and in particular in MOVA, we consider sets $S$ that interpolates in a unique group homomorphism.

Now we can describe the interactive protocols that are used in MOVA. 

\subsubsection{GHID Problem interactive proof}

We first describe the interactive proof for the GHID Problem.
Consider $G,H$ and $S$ as defined above where $|H| = d$. 
Consider a security parameter $l$. In this protocol, a prover convinces a verifier that $S$ interpolates in a group homomorphism $\hom$.
$\hom$ the secret witness known by the prover. 
The protocol is described in Figure~\ref{fig:ghiproof}.


\begin{figure}[!h]
    \centering
    \[
        \begin{split}
        &\mbox{\bf GHIproof}_l(S)\\
        &\mbox{\bf Parameters: } G,H,d=|H|\\
        &\mbox{\bf Input: } l,S = \{(g_i,e_i)|i=1,...,s\} \subseteq G\times H\\
        \end{split}
    \]
    \[
        \begin{array}{ccc}
        \hline
            \mbox{Prover } P                      &                  & \mbox{Verifier } V                                  \\
        \hline
                                                  &                  &                                                     \\
                                                  &                  & r_i \in_U G, a_{i,j} \in_U \XdX{\Z}{d}  \\ 
                                                  &                  & \mbox{compute}                                      \\
                                                  &                  & u_i = dr_i + \sum_{j=1}^{s}{a_{i,j}g_{j}}           \\
                                                  & \spleftarrow{u_i} & w_i = \sum_{j=1}^{s}{a_{i,j}e_{j}}           \\
                                                  &                  & i = 1,...,l \mbox{ and } j=1,...,s      \\
                                                  &&\\
            v_i = \hom(u_i) \quad i=1,...,l & \sprightarrow{\commit{v_1,...,v_l}} &                                   \\
                                                  &&\\
                                                  & \spleftarrow{r_i,a_{i,j}}           &                                   \\ 
                                                  &&\\
             \mbox{check } u_i = dr_i + \sum_{j=1}^{s}{a_{i,j}g_{j}}& \sprightarrow{\mathrm{open}(v_1,....,v_l)}  &        \\
                                                  &&\\
                                                  &                                    & \mbox{check } v_i = w_i           \\
        \hline
        \end{array}
    \]
    \caption{$\mathrm{GHIproof}_{l}(S)$}
    \label{fig:ghiproof}
\end{figure}

\subsubsection{coGHID Problem interactive proof}
Let us describe the interactive proof for the coGHID Problem.
Consider $G,H$,$S$ and $T$ as defined above where $|H| = d$ and $p$ is the smallest prime factor of $d$. Consider also a security parameter $l$. 
In this protocol, the prover convinces the verifier that for at least one $(x_i,z_i) \in T$
we have that $z_i \neq \hom(x_i)$, where $\hom$ uniquely interpolates $S$.
The protocol is depicted in Figure~\ref{fig:coghiproof}. 

One might notice that at step 2 of the protocol, when the prover has to find $\lambda_i$, for $i = 1,...,l$, this is equivalent to solve the discrete logarithm problem in $H$.
But $\lambda_i \in \XdX{\Z}{p}$, so in practice we choose $d$ such that its smallest prime factor is not too large.

\begin{figure}[!h]
    \hskip -25pt
    \[
        \begin{split}
        &\mbox{\bf coGHIproof}_l(S,T)\\
        &\mbox{\bf Parameters: } G,H,d=|H|,p \mbox{ smallest prime factor of } d\\
        &\mbox{\bf Input: } l,S = \{(g_i,e_i)|i=1,...,s\} \subseteq G\times H,\\
        & \quad\quad T = \{(x_i,z_i)|i=1,...,t\} \subseteq G \times H\\  
        \end{split}
    \]
    \[
        \begin{array}{ccc}
        \hline
            \mbox{Prover } P                      &                  & \mbox{Verifier } V                                  \\
        \hline
                                                  &                  &                                                     \\
                                                  &                  & r_{i,k} \in_U G, a_{i,j,k} \in_U \XdX{\Z}{d}  \\ 
                                                  && \mbox{ and } \lambda_i \in_U \XdX{\Z}{p} \\
                                                  &                  & \mbox{compute}                                      \\
                                                  &                  & u_{i,k} = dr_{i,k} + \sum_{j=1}^{s}{a_{i,j,k}g_{j}} + \lambda_i x_i           \\
                                                  & \spleftarrow{u_{i,k},w_{i,k}} & w_{i,k} = \sum_{j=1}^{s}{a_{i,j,k}e_{j}} + \lambda_i z_i           \\
                                                  &                  & \mbox{for } i = 1,...,l \mbox{ , } j=1,...,s       \\
                                                  & & \mbox{ and } k=1,...,t\\
                                                  &&\\
        v_{i,k} = \hom(u_{i,k}),  y_k = \hom(x_k) &&\\
     \mbox{for } i=1,...,l \mbox{ and } k=1,...,t &&\\
     \mbox{find } \lambda_i \mbox{ st } w_{i,k} -v_{i,k} = \lambda_i(z_k-y_k) & \sprightarrow{\commit{\lambda_1,...,\lambda_l}} &                                   \\
                                                  &&\\
                                                  & \spleftarrow{r_{i,k},a_{i,j,k}}           &                                   \\ 
                                                    &&\\
                                       \mbox{check} &&\\
        u_{i,k} = dr_{i,k} + \sum_{j=1}^{s}{a_{i,j,k}g_{j}} + \lambda_i x_i &&\\
           w_{i,k} = \sum_{j=1}^{s}{a_{i,j,k}e_{j}} + \lambda_i z_i   & \sprightarrow{\mathrm{open}(\lambda_1,....,\lambda_l)}  &                                   \\
                                                  &&\\
                                                  &                                    & \mbox{check that $\lambda_i$ are correct}           \\
        \hline
        \end{array}
    \]
    \caption{$\mathrm{coGHIproof}_{l}(S)$}
    \label{fig:coghiproof}
\end{figure}


\subsection{Advantages of MOVA}
MOVA has some advantages, when comparing it to classical signature schemes. To begin, MOVA scheme is an undeniable scheme. Therefore
verification can be done only with the help of the signer whereas classical signature schemes are universally verifiable (everyone can verify the signature).
In some application where privacy is an issue, this property can be useful. For example we can think of an application where someone wants to sign a document,
but he does not want that anyone can see that he signed it.

The other main advantage of MOVA is that we can have very short signatures (for example 20 bits as suggested in~\cite{cite:thesis-monnerat}).
Short signature might be very useful when dealing with mobile or resource limited devices. For example, we could encode some information with his signature 
in a QR-code.
