\chapter{The Application}
\label{chap:application}

    Before starting the development, the first step was to find an application on mobile devices 
    that has to use MOVA without being too 
    artificial. Different applications were considered and compared to choose the most adapted. 
    The application we came to is named {\bf University Contest}. The shortness of the signature and batch verification were the two main benefits of MOVA. 
    In this chapter, the developed application will be presented. 
    Then we will establish a threat model and say why MOVA is more appropriated than classical signature in our mobile application.

    
    \section{Overview}
    \label{sec:app-overview}
    We consider an application where several universities are participating to a tournament.
    We can imagine that for the first round of the tournament, universities are organized into
    pairs. Let us consider one pair, and let us denote the two universities by $U_1$ and $U_2$. $U_1$ and $U_2$  are in competition.
    In this application, $U_1$ provides quizzes/challenges/riddles to students in $U_2$. Similarly $U_2$ provides quizzes to students in $U_1$.
    Students in a university are forming teams. We assume that teams have a certain amount of time (for example one week) to complete the current quiz/challenge/riddle. 
    Each team in a university have a different quiz\footnote{If several teams in a university have the same quiz, then they can copy on each other to increase the final score.}.
    When completed they send it to the server of both universities and get a score. Then the university who provided the quiz also provides the solution of the quiz and the score the 
    team did, to the students. At the end of the first round, all the best of the two universities 
    (i.e. the ones with the highest cumulated score of all teams in the university) in the pairs are 
    promoted to the second round. The game goes on the same way until the final round, where only two universities are remaining. The best of the two wins the game. 

    \begin{figure}[!h]
        
        \centering
        \subfloat[$S_1$ sending quizzes to teams in $U_2$]{
            \begin{pspicture}(1,0)(4,5)
                \psellipse[fillcolor=gray!34,fillstyle=solid,linecolor=gray!34](1,2.5)(0.7,2)
                \psellipse[fillcolor=gray!34,fillstyle=solid,linecolor=gray!34](4,2.5)(0.7,2)
                \rput(1,4){\color{gray}$U_1$}
                \rput(4,4){\color{gray}$U_2$}
                \rput(1,3){$S_1$}
                \rput(4,3){$S_2$}
                \rput(4,1){$T_i$}
                \psline{->}(1.3,2.7)(3.7,1.3)
                \rput(2.5,2.5){$(Q,\sigma)$}
            \end{pspicture}
        }

        \subfloat[$T_i$ sending back the filled in quiz]{
            \begin{pspicture}(1,0)(4,5)
                \psellipse[fillcolor=gray!34,fillstyle=solid,linecolor=gray!34](1,2.5)(0.7,2)
                \psellipse[fillcolor=gray!34,fillstyle=solid,linecolor=gray!34](4,2.5)(0.7,2)
                \rput(1,4){\color{gray}$U_1$}
                \rput(4,4){\color{gray}$U_2$}
                \rput(1,3){$S_1$}
                \rput(4,3){$S_2$}
                \rput(4,1){$T_i$}
                \psline{->}(4.1,1.3)(4.1,2.7)
                \rput(4.3,2){\tiny $\tilde{Q}$}
                \psline{<-}(3.9,1.3)(3.9,2.7)
                \rput(3.5,2){\tiny $(\tilde{Q},\tilde{\sigma})$}
                \psline{<-}(1.3,2.7)(3.7,1.3)
                \rput(2.4,1.6){\tiny $(\tilde{Q},\tilde{\sigma})$}
            \end{pspicture}
        }
        
        \caption{Application Overview}
    
    \end{figure}

    \section{Security Overview}
    \label{sec:app-sec}
    For this part let us consider one pair with universities $U_1$ and $U_2$ as before. Let us denote the server of $U_1$ by $S_1$ and the server of $U_2$ by $S_2$.
    We can assume that communication between $S_1$ and $S_2$ are confidential, authenticated and preserve integrity.
    When $U_1$ creates a quiz/challenge for students of $U_2$ they sign it with MOVA and then send it to them. In that way, the students are ensured that the quiz comes indeed
    from $U_1$ and that it is not a fake quiz from other cheating students who wants to make them lose time, hence lose score. Furthermore, only the designated team can verify the signature.
    Then, when the students are sending back the quiz to $S_2$, $S_2$ can sign the filled in quiz with MOVA and send it to back to the team. The team can verify that the signature 
    is correct\footnote{In that model, we consider that the students do not trust the server and want to be sure that answer were not changed or deleted.},
    and send it to $S_1$. Finally $S_1$ receives the filled in quiz and verify both signatures.
    For signatures coming from $S_2$, $S_1$ can run a batch verification with $S_2$.
    Then when $S_1$ provides the solutions with score the team, he signs everything and sends it across. Therefore, students are ensured that the solution and the score are correct.

    When a team is starting to communicate with a server, the team is authenticated using a preshared secret. We can assume that when a team registers, it has to choose a password.

    \section{Threat Model}
    \label{sec:threat}
    Let us now describe the threat model, by showing what an adversary could do or what he might want to do if he is also a participant for example.
    \subsubsection{Fake Quizzes}
    The first thing that an adversary might do is to forge fake quizzes. Hence the use of the signature when sending new quizzes. A reason why an adversary might do that is  to
    make the teams fill in useless quizzes. In that case, the adversary might be another team from the opposing university. Another reason might be to make university servers receive 
    useless filled in quizzes.
    \subsubsection{Fill Another Team's Quiz}
    An adversary might want to send a quiz assigned to a team he is not belonging to, so that the team gets a zero score on that quiz. 
    But by construction of the application, he cannot steal a quiz and make the score count for his team. Even if he is also participating to the contest
    because quizzes are unique and associated to their respective teams.
    \subsubsection{Modify the Answer in a Filled in Quiz}
    Another type of attack one could do is when the team sends the filled in quiz, the adversary could modify what is sent so that the team loses score. 
    Therefore, as stated previously, the team sends first the filled in quiz to his university server so that it is signed. Then the team can verify 
    that nothing was modified. 

    \section{Problems}
    \label{sec:app-problem}
    We need to authenticate the students to their university server. Because if this is not the case, then anyone can forge a valid signature. 
    When sending the filled in quiz 
    to the university to get the signature, if no authentication is provided then anyone can make the server sign anything\footnote{This server only sign.
    He does not do any verification because all verifications are done on the other server (the one which will correct the quiz).}.


    \section{Using Classical Signatures}
    \label{sec:app-classic}
    In this application we could try to use classical signature. For the signature of the unfilled quiz (the signature signed by $S_1$ when $S_1$ is the quiz provider), 
    MOVA could bring something compared to classical signatures, due to the following reasons:
    \begin{itemize}
        \item We use that MOVA is not universally verifiable, because only the designated team must verify the signature. 
        \item For the signature of the filled in quiz (signed by $S_2$, when $S_1$ is the quiz provider), we exploit the fact that $S_1$
            runs batch verification. For example at the end of the day, the server has received one quiz per team, and the signatures have to be 
            verified. So all signatures are verified in one shot. We can assume that verification is scheduled at the end of each day or the end of a round. 
        \item The fact that the signatures are very short is always an advantage when dealing with mobile devices. Furthermore since the server has to
            store some signature to run the batch verification, we gain some storage-space by having shorter signature. We could schedule the batch verification
            at the end of each month for example.
    \end{itemize}
    


