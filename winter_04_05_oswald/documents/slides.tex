\documentclass[a4, landscape, slidesonly]{seminar}
\usepackage{hyperref,amssymb, amsfonts, amsthm, amsmath, algorithm, algorithmic, verbatim, epsfig, multicol}
\usepackage{psfig}
\usepackage{latexsym}
\usepackage{color}
\slideframe{none}


 \setlength{\parindent}{0pt} 

% color definitions
\definecolor{titlecolor}{rgb}{.6,.2,0}
\definecolor{emphcolor}{rgb}{.6,.2,0}
\renewcommand{\emph}[1]{{\color{emphcolor} #1}}
\newcommand{\slidetitle}[1]{{\large \color{titlecolor} #1}}

% font definitions
\renewcommand{\slidefonts}{\sf}
\newcommand{\logo}[0]{\begin{center}\epsfig{file=logo_epfl_coul.eps, width=1.5cm} ~~~~~~~~~~~~~~~~~~~~~~~~~~~~~~~~~~~~~~~~~~~~~~~~~~~~~~~~~~ \epsfig{file=logo_lasec_coul.eps, width=1.5cm}\end{center}}
\newcommand{\supervisor}[4]{\begin{center}\begin{normalsize}{\bf #1}\\#2\\#4\end{normalsize}\end{center}}

\begin{document}
\begin{slide}
 \begin{center}
   {\Large \textbf{Generic Homomorphic Undeniable Signature Scheme: Optimizations}}\\[1cm]
%Titel anpassen
   \emph{Yvonne Anne Oswald}\\
   
   Semester Project February 2005\\[1cm]
  
\begin{tabular}{cc}
\begin{tabular}{p{4.0cm}}
\supervisor{Responsible}{Prof. Serge Vaudenay}{serge.vaudenay@epfl.ch}{}
\end{tabular}&
\begin{tabular}{p{4.0cm}}
\supervisor{Supervisor}{Jean Monnerat}{jean.monnerat@epfl.ch}{}
\end{tabular}
\end{tabular}
\end{center}
\hrule
\logo

\end{slide}

%%%%%%%%%%%%%%%%%%%%%%%%%%%%%%%%%%%%%%%%%%%%%%%%%%%%%%
\begin{slide}
\slidetitle{Outline}
\\[.2cm]\begin{itemize}
\item Introduction generic homomorphic undeniable signatures
\item Homomorphisms: quartic residue symbol, Jacobi symbol, discrete logarithm, RSA
\item Comparison signature generation
\item Conclusion
\end{itemize}
\end{slide}
%%%%%%%%%%%%%%%%%%%%%%%%%%%%%%%%%%%%%%%%%%%%%%%%%%%%%%
\begin{slide}
\slidetitle{Undeniable Signatures}
\\[.2cm]
\begin{itemize}
\item Verification of validity of signature only by interaction with Signer
\item Complete signature scheme consists of 
\begin{itemize}
\item Key generation algorithm
\item Signature algorithm
\item Confirmation protocol
\item Denial Protocol
\end{itemize}
\end{itemize}
\end{slide}
%%%%%%%%%%%%%%%%%%%%%%%%%%%%%%%%%%%%%%%%%%%%%%%%%%%%%%
\begin{slide}
\slidetitle{GHI Problem}
\\[.2cm]
\emph{Definition:}
$G, H$ Abelian groups, $S := \{(x_{1},y_{1}),\ldots,(x_{s},y_{s})\}$ $ \subseteq G\times H$. We say that $S$ \emph{interpolates in a group homomorphism} 
 if there exists a homomorphism $\phi : G \to H$ such that $\phi (x_{i}) = y_{i}$ for $i = 1, \ldots, s$.\\[0.3cm]
\emph{GHI Problem \big(Group Homomorphism Interpolation Problem\big)}
\emph{Parameters:} two Abelian Groups $G$ and $H$, a set of $s$ points $S \subseteq G \times H$. \\
\emph{Input:} $x \in G$\\ 
\emph {Problem:} find $y \in H$ such that $\big( x,y\big)$ interpolates with $S$ in a group 
 homomorphism.\\[0.3cm] 
The GHI problem is the generalization of many cryptographic problem: discrete logarithm, Diffie-Hellmann, RSA.
\end{slide}
%%%%%%%%%%%%%%%%%%%%%%%%%%%%%%%%%%%%%%%%%%%%%%%%%%%%%%
\begin{slide}
\slidetitle{Generic homomorphic undeniable signatures}
\\[0.5cm]
\emph{Key generation:}
\begin{itemize}
\item Select $G, H$ Abelian groups and hom. $\phi : G \to H$ (private key)
\item Compute public key $K := \{(x_{key1},\phi(x_{key1}),\ldots,(x_{keyk},\phi(x_{keyk}))\}$, $x_i$ generated from seed $\rho$ using det. pseudorandom generator
\end{itemize}
\emph{Signature and protocols:}
\begin{itemize}
\item Generate $(x_1, \ldots, x_s)$ from message $m$ using det. pseudorandom generator
\item Compute signature $S:= \{(x_{sig1},\phi(x_{sig1}),\ldots,(x_{sigs},\phi(x_{sigs}))\}$
\item Confirmation/denial protocol: proving/disproving that $K$ interpolates with $S$ in a homomorphism
 \end{itemize} 
\end{slide}
%%%%%%%%%%%%%%%%%%%%%%%%%%%%%%%%%%%%%%%%%%%%%%%%%%%%%%%%%
\begin{slide}
\slidetitle{Generic homomorphic undeniable signatures}
\\[.2cm]
\begin{itemize}
\item Security of generic homomorphic undeniable signatures depends on hardness of GHI problem
\item Various homomorphisms suitable (hard characters, discrete logarithm, RSA exponentiation)
\item Summer 2004: Demonstrator for MOVA signature scheme using quartic residue symbol as homomorphism
\end{itemize}
\end{slide}
%%%%%%%%%%%%%%%%%%%%%%%%%%%%%%%%%%%%%%%%%%%%%%%%%%%%%%%%%
\begin{slide}
\slidetitle{Description of project}
\\[.2cm]
\begin{itemize}
\item Optimize existing implementation of quartic residue symbol
\item Implement 3 additional homomorphisms (Jacobi symbol, discrete logarithm, RSA exponentiation) and compare them to each other
\item Programming language C, GNU Multiple Precision Arithmetic Library GMP
\end{itemize}
\end{slide}
%%%%%%%%%%%%%%%%%%%%%%%%%%%%%%%%%%%%%%%%%%%%%%%%%%%%%%
\begin{slide}
\slidetitle{Quartic residue symbol $\mathbf{\chi}_{\beta}(\alpha)$}
\\[0.5cm]
\emph{Definition}
Let $\alpha,\beta \in \mathbb{Z}[{i}]$ be such that $(1+i) \nmid \beta$ and $\gcd(\beta,\alpha) = 1$. \\
The \emph{quartic residue symbol} is defined as $ 
 \mathbf{\chi}_{\beta} : \mathbb{Z}[{i}] \to \{ 0,\pm 1, \pm i\}$ 
 \begin{displaymath} 
 \mathbf{\chi}_{\beta}(\alpha) = \left\{ 
 \begin{array}{ll} 
     \big(\alpha^{\frac{N(\beta)-1}{4}}\big)\bmod \beta & 
         \textrm{if $\beta$ prime}\\ 
     \prod _{i} ~\mathbf{\chi}_{\beta _{i}}(\alpha) & 
         \textrm{if $\beta = \prod _ {i} \beta _{i},~ \beta_{i}$ prime} 
 \end{array} \right. 
 \end{displaymath}
 
 \emph{Properties}
  \begin{itemize} 
 \item[-] Modularity: $\chi_{\beta}(\alpha) = \chi_{\beta}(\alpha \bmod \beta)$ 
 \item[-] Multiplicativity: $\chi_{\beta}(\alpha\alpha') = \chi_{\beta}(\alpha)\chi_{\beta}(\alpha')$ 
 \item[-] Reciprocity Law: if $\alpha,~\beta$ primary:\\ $~~~~~~~~~~~~~\chi_{\beta}(\alpha) = \chi_{\alpha}(\beta)\cdot 
     (-1)^{\frac{N(\alpha)-1}{4}\cdot \frac{N(\beta)-1}{4}}~~~$ 
 \end{itemize} 
\end{slide}
%%%%%%%%%%%%%%%%%%%%%%%%%%%%%%%%%%%%%%%%%%%%%%%%%%%%%%
\begin{slide}
\slidetitle{Quartic residue symbol $\mathbf{\chi}_{\beta}(\alpha)$}
\\[0.5cm]
\emph{Basic algorithm}\\[.2cm]
$t \gets 0$\\
{WHILE} $(N(\alpha) > 1)$ DO\begin{itemize}
\item[] $\alpha \gets \alpha \bmod \beta~$~~~~~~~~~~~~~~~~~~~~~~~~~~~~~~~~~~~~(Modularity)
\item[] find $\alpha'$ primary such that $i^j \cdot (1+i)^k \cdot \alpha'$
\item[] set $\alpha \gets \alpha'$ and adjust $t$~~~~~~~~~~~~~~~~~~~~~~~~~(Multiplicativity)
\item[] swap $\alpha$ and $\beta$ and adjust $t$~~~~~~~~~~~~~~~~~~~~~(Reciprocity)
\end{itemize} 
{IF} $(N(\alpha) = 1)$ return $i^t$
\end{slide}
%%%%%%%%%%%%%%%%%%%%%%%%%%%%%%%%%%%%%%%%%%%%%%%%%%%%%%%%%%%
\begin{slide}
\slidetitle{Quartic residue symbol $\mathbf{\chi}$}
\\[0.5cm]
\emph{Damg\aa rd's algorithm}\\[0.2cm]
$t \gets 0$\\
{WHILE} $(\alpha \ne \beta)$ DO\begin{itemize}
\item[] $\alpha \gets \alpha - \beta~$
\item[] find $\alpha'$ primary such that $i^j \cdot (1+i)^k \cdot \alpha'$
\item[] $\alpha \gets \alpha'$ and adjust $t$~~~~~~~~~~~~~~~~~~~~~~~~~(Multiplicativity)
\item[] {IF} $(N(\alpha) > N(\beta))$ \begin{itemize}
\item[] swap $\alpha$ and $\beta$ and adjust $t$~~~~~~~~~~~~~(Reciprocity)
\end{itemize} 
\end{itemize}
{IF} $(\alpha = 1)$ return $i^t$
\end{slide}
%%%%%%%%%%%%%%%%%%%%%%%%%%%%%%%%%%%%%%%%%%%%%%%%%%%%%%
\begin{slide}
\slidetitle{Quartic residue symbol $\mathbf{\chi}$}
\\[0.8cm]
\emph{Implementation}\\[.2cm]
Basic functions in $\mathbb{Z}[i]$  \begin{itemize} 
 \item Multiplication: $\alpha \cdot \beta$ 
 \item Modulo: $\alpha \bmod \beta$ 
 \item Norm: $N(\alpha)$ 
 \item Division by $(1+i)^r$ 
 \item Primarization: transforms $\alpha$ into its primary associate if possible 
 \end{itemize}

\end{slide}
%%%%%%%%%%%%%%%%%%%%%%%%%%%%%%%%%%%%%%%%%%%%%%%%%%%%%%
\begin{slide}
\slidetitle{Quartic residue symbol $\mathbf{\chi}$}
\\[0.8cm]
\emph{Optimization}
\begin{itemize} 
\item Scrutinize every line of code 
\item Implement faster algorithms
\item Remove unnecessary function calls
\item Use faster, more sophisticated GMP functions
\item Reduce the number of GMP variables 
\item Examine different implementation variants
\item Apply general optimization techniques \\
 \end{itemize}
 Number of lines of code: not optimized 1162, optimized 603
\end{slide}
\begin{slide}
\slidetitle{Division by $(1+i)^r$}
\\[.5cm]
 \[\frac{\alpha }{(1+i)} = \frac{Re(\alpha)+Im(\alpha)}{2} + \frac{Im(\alpha)- Re(\alpha)}{2} i\]
\[\frac{\alpha}{(1+i)^r} =  \frac{i^{3k} \Big(\frac{Re(\alpha)}{2{^k}} + \frac{Im(\alpha)}{2{^k}}i\Big) }{(1+i)^b}~~,~~r = 2k + b\]
\setlength{\columnseprule}{0pt}
\setlength{\columnsep}{.5cm}
\begin{multicols}{2}
Not optimized: 
\begin{itemize} 
\item Compute $(1+i)^r$ 
\item Divide $\alpha$ by result\\[.5cm]
 \end{itemize}
Optimized: 
\begin{itemize} 
\item Rightshift $Re(\alpha)$ and $Im(\alpha)$ by $k+b$ considering $3k$
\item If $b=1$ perform necessary addition and subtraction
 \end{itemize}
\end{multicols}
\end{slide}
%%%%%%%%%%%%%%%%%%%%%%%%%%%%%%%%%%%%%%%%%%%%%%%%%%%%%%
\begin{slide}
\slidetitle{Quartic residue symbol $\mathbf{\chi}$}
\\[0.8cm]
\emph{Tests}
\begin{itemize}
\item Time measurements under Windows and Linux
\item Windows platform Intel(R)4 1.4 GHz Desktop Computer,\\ 256 MB RAM, Windows XP
\item Average values produced by test series of 1000 tests
\item 512 bit random numbers generated with a GMP function \\
Gaussian integers: real and imaginary part 512 bit random numbers 
\end{itemize}
\end{slide}
%%%%%%%%%%%%%%%%%%%%%%%%%%%%%%%%%%%%%%%%%%%%%%%%%%%%%%
\begin{slide}
\slidetitle{Quartic residue symbol $\mathbf{\chi}$}
\\[0.8cm]
\emph{Results subfunctions}\\
 \begin{table}[htb] 
  \begin{tabular}{|l c c c|} 
 \hline 
  Running time (in ms)&  {not optimized} &  {optimized} & {GMP (in $\mathbb{Z}$)} \\ 
 \hline  
 Multiplication in $\mathbb{Z}[i]$ & 0.078  & 0.049 & 0.010\\
 Modulo in $\mathbb{Z}[i]$ &  0.141 & 0.104& 0.001\\ 
 Division by $(1+i)^r$  &  0.061 & 0.015 & 0.001\\ 
 Primarization &  0.071& 0.006 & \\ 
 \hline 
 \end{tabular}
 \end{table} 
 
\end{slide}
%%%%%%%%%%%%%%%%%%%%%%%%%%%%%%%%%%%%%%%%%%%%%%%%%%%%%%
\begin{slide}
\slidetitle{Quartic residue symbol $\mathbf{\chi}$}
\\[0.8cm]
\emph{Results quartic residue symbol}\\
 \begin{table}[htb] 
 \begin{tabular}{|l c c|} 
 \hline 
 Running time and iterations &  {time in ms} &  {iterations}\\ 
 \hline  
 Basic algorithm not optimized & 62.79 & 249.27\\
 Basic algorithm optimized & 31.57 & 249.27\\
 Damg\aa rd's algorithm  &  24.22 & 512.84\\
 \hline 
 \end{tabular}
 \end{table}  
\end{slide}
%%%%%%%%%%%%%%%%%%%%%%%%%%%%%%%%%%%%%%%%%%%%%%%%%%%%%%
\begin{slide}
\slidetitle{Quartic residue symbol $\mathbf{\chi}$}
\\[0.8cm]
\emph{MOVA signature scheme}
\begin{itemize}
\item $n = pq$ where $p,~q$ rational primes, $p\equiv q\equiv 1 \pmod 4$
\item Find $\pi,\sigma$ such that $p=\pi\bar{\pi}$ and $q = \sigma\bar{\sigma}$ with algorithms by Tonelli and Cornacchia
\item Select $G := \mathbb{Z}[i]/\beta\mathbb{Z}[i],~ G \cong \mathbb{Z}_n^*$ with $\beta= \pi \sigma$, $G \cong \mathbb{Z}_p^*$ with $\beta= \pi$
\end{itemize}
\end{slide}
%%%%%%%%%%%%%%%%%%%%%%%%%%%%%%%%%%%%%%%%%%%%%%5
\begin{slide}
\slidetitle{Quartic residue symbol $\mathbf{\chi}$}
\\[0.8cm]
\emph{Results MOVA Setup}\\
 \begin{table}[htb]  
 \begin{tabular}{|l c c|} 
 \hline 
Running time and iterations&  {time in ms} &  {iterations}\\ 
 \hline  
 $\beta= \pi \sigma$ & &\\
 \hline
 Basic algorithm optimized & 32.12 & 248.81\\
 Damg\aa rd's algorithm  &  50.63 & 766.12\\
 \hline  
 $\beta= \pi$ & &\\
 \hline
 Basic algorithm optimized & 14.31 & 2123.87\\
 Damg\aa rd's algorithm  &  38.59 & 640.71\\
 \hline 
 \end{tabular}
 \end{table}  
\end{slide}
%%%%%%%%%%%%%%%%%%%%%%%%%%%%%%%%%%%%%%%%%%%%%%%%%%%%%%
\begin{slide}
\slidetitle{Quartic residue symbol $\mathbf{\chi}$}
\\[.5cm]

%Damg\aa rd's algorithm slower because 

\emph{Mixed Algorithm}
\begin{itemize}
\item[] $\alpha \gets \alpha \bmod \beta~$
\item[] return Damg\aa rd$(\alpha,\beta)$
\end{itemize}
\end{slide}
%%%%%%%%%%%%%%%%%%%%%%%%%%%%%%%%%%%%%%%%%%%%%%%%%%%%%
\begin{slide}
\slidetitle{Quartic residue symbol $\mathbf{\chi}$}
\\[0.7cm]
\emph{Results including mixed algorithm}
 \begin{table}[htb] 
 \begin{tabular}{|l c c|} 
 \hline 
MOVA setup running time and iterations &  {time in ms} &  {iterations}\\ 
  \hline  
 $\beta= \pi \sigma$ & &\\
 \hline  
 Basic algorithm optimized & 32.12 & 248.81\\
 Damg\aa rd's algorithm  &  50.63 & 766.12\\
 Mixed algorithm & 24.65 & 511.92\\
 \hline 
 $\beta= \pi$ & &\\
 \hline
 Basic algorithm optimized & 14.31 & 2123.87\\
 Damg\aa rd's algorithm  &  38.59 & 640.71\\
 Mixed algorithm & 9.03 & 255.95\\
 \hline 
 \end{tabular}
 \end{table}
\end{slide}
%%%%%%%%%%%%%%%%%%%%%%%%%%%%%%%%%%%%%%%%%%%%%%%%%%%%%%
\begin{slide}
\slidetitle{Jacobi Symbol $(\frac{a}{m})$}
\\[0.3cm]
\begin{itemize}
\item Jacobi symbol equivalent of quartic residue symbol in $\mathbb{Z}$
\item Implement basic algorithm (modulo for reduction)
\item Compare with GMP function \ttfamily mpz\_jacobi \sffamily (binary algorithm)
\item MOVA setup: $n=pq, ~p,q$ prime, $a \in \mathbb{Z}^*_n, m = p$\\
\end{itemize}
\emph{Results Jacobi symbol}
 \begin{table}[htb] 
 \begin{tabular}{|l c c|} 
 \hline 
Running time and iterations  &  {time in ms} &  {iterations}\\ 
 \hline  
 Basic algorithm & 1.261 & 187.71\\
 Binary algorithm (GMP) & 0.116 & \\
 \hline 
 \end{tabular}
 \end{table}
\end{slide}
%%%%%%%%%%%%%%%%%%%%%%%%%%%%%%%%%%%%%%%%%%%%%%%%%%%%%%
\begin{slide}
\slidetitle{Homomorphism based on discrete logarithm}
\\[0.5cm]
\begin{itemize}
\item $n = pq$ with $p = rd + 1$, $q$, $d$ prime, $\gcd(q-1,d)=1,~\gcd(r,d)= 1$
\item $g$ generating a subgroup of $\mathbb{Z}^{*}_{p}$ of order $d$ \\
\end{itemize}
Construct a homomorphism suitable for the generic homomorphic signature scheme by computing a discrete logarithm in a small subgroup of $\mathbb{Z}^{*}_{n}$ like this: 
 \begin{displaymath} 
     \phi : \mathbb{Z}^{*}_{n} \to \mathbb{Z}_{d} ~~~\phi (x) = \log_g(x^r \bmod~ p) 
 \end{displaymath} \\ 
GMP provides a function for exponentiation modulo an integer\\
$\longrightarrow$ only discrete logarithm to implement
 
\end{slide}
%%%%%%%%%%%%%%%%%%%%%%%%%%%%%%%%%%%%%%%%%%%%%%%%%%%%%%
\begin{slide}
\slidetitle{Homomorphism based on discrete logarithm}
\\[0.8cm]
\emph{Precomputed table}\\
Construct table containing discrete logarithm for all elements\\
Size of table: $2^{20}$ 20 bit integers, key 512 bits long 
\begin{itemize}
\item Hash table: array of integers (32 bits)
\item Key management: key not saved, index into array
\item Collision management: 24 LSB of key, key $>$$>$ 24, \ldots
\item Insertion/Retrieval methods: collision check, correctness check
\item Preprocessing time: $O(d)$, running time: $O(1)$
\end{itemize} 
\end{slide}
%%%%%%%%%%%%%%%%%%%%%%%%%%%%%%%%%%%%%%%%%%%%%%%%%%%%%%
\begin{slide}
\slidetitle{Homomorphism based on discrete logarithm}
\\[0.8cm]
\emph{Baby step giant step algorithm (BSGS)}
\begin{itemize}
\item Store $O(\sqrt d)$  elements in hashtable (baby steps)
\item Loop from 0 to $O(\sqrt d)$ to find logarithm (giant steps)
\item Use hash table from previous algorithm, less collisions
\item Preprocessing time: $O(\sqrt d)$, running time: $O(\sqrt d)$
\end{itemize}
\begin{comment}
Compute $x = \log_g(y)$\begin{itemize}
\item[] Set $m \gets \lceil \sqrt d \rceil$ 
\item[] Construct a hash table with entries $(g^j,j)$ for $0 \le j \le m$ 
\item[] FOR $i$ from $0$ to $m-1$ DO 
	\begin{itemize} 
 	\item[] Check if there is an entry $j$ for $y$ in hash table
 	\item[] IF $(\gamma = g^j)$ return $x \gets im + j$ 
 	\item[] Set $y \gets y ^{-m}$
 	\end{itemize}
\item[] END FOR
\end{itemize}
\end{comment}
\end{slide}
%%%%%%%%%%%%%%%%%%%%%%%%%%%%%%%%%%%%%%%%%%%%%%%%%%%%%%
\begin{slide}
\slidetitle{Homomorphism based on discrete logarithm}
\\[0.5cm]
\emph{Pollard's rho algorithm}\begin{itemize}
\item Group G partitioned into three sets $S_0, S_1, S_2$ of roughly  equal size
\item Define sequence of group elements $x_i$ and integers $a_i, b_i$  satisfying \emph{$x_i = g^{a_i}y^{b_i}$} $\forall i \ge 0$ by $x_0 = 1, ~a_0 = 0, ~b_0 = 0$ and  \[x_{i+1} = \left\{\begin{array}{lr} 
         y x_i \bmod p & ~~~~~\textrm{if } x_i \in S_0\\ 
         ~x_i^2 \bmod p & ~~~~~\textrm{if } x_i \in S_1\\ 
         g x_i \bmod p & ~~~~~\textrm{if } x_i \in S_2\\ 
     \end{array} \right. \] 
\item Compute iteratively $x_i$ and $x_{2i}$ until $x_i = x_{2i}$
\item Derive logarithm from representation $x_i = g^{a_i}y^{b_i}$
\item Running time: $O(\sqrt d)$    
\end{itemize}  
\end{slide}
%%%%%%%%%%%%%%%%%%%%%%%%%%%%%%%%%%%%%%%%%%%%%%%%%%%%%%
\begin{slide}
\slidetitle{Homomorphism based on discrete logarithm}
\\[0.5cm]
\emph{Results}
 \begin{table}[htb] 
 \begin{tabular}{|l c c|} 
 \hline 
Table construction  &  {time in s} &  {collisions}\\ 
 \hline  
 Precomputed table & 16.616 & 14 274\\
 Baby step giant step & 6.023 & 0 \\
 \hline 
 & &  \\
Running time and iterations  &  {time in ms} &  {iterations}\\ 
 \hline   
 GMP \ttfamily mpz\_powm \sffamily & 9.44 & ~~~~~~\\
 \hline
 Precomputed table & 9.66 & 1.04\\
 Baby step giant step & 19.47 & 388.36 \\
 Pollard's rho & 74.92 & 1037.49 \\
 \hline 
 \end{tabular}
 \end{table}
\end{slide}
%%%%%%%%%%%%%%%%%%%%%%%%%%%%%%%%%%%%%%%%%%%%%%%%%%%%%%
\begin{slide}
\slidetitle{RSA}
\\[0.2cm]
\begin{itemize}
\item Select $n=pq,~ p,q$ prime, $\phi(n) = (p-1)(q-1)$
\item Choose $e$ such that $1 < e < \phi(n)$ and $ \gcd(e,\phi(n))=1$ 
\item Compute $ d=e^{-1}\bmod \phi(n)$ \\
\end{itemize}

Homomorphism $\phi$ suitable for generic homomorphic signature scheme:
 \begin{displaymath} 
     \phi : \mathbb{Z}^{*}_{n} \to \mathbb{Z}^*_{n} ~~~\phi (x) = (x^d \bmod~ n) 
 \end{displaymath}
\emph{Results}
 \begin{table}[htb] 
 \begin{tabular}{|l c|} 
 \hline 
Running time&  {time in ms}\\ 
 \hline  
 RSA exponentiation ~~~~~~~~~~~& ~~33.87\\
 \hline 
 \end{tabular}
 \end{table}
\end{slide}
%%%%%%%%%%%%%%%%%%%%%%%%%%%%%%%%%%%%%%%%%%%%%%%%%%%%%%
\begin{slide}
\slidetitle{Signature Generation}
\\[1.0cm] 
\emph{Security parameters}
\begin{itemize}
\item Quartic residue symbol $n$ 1024 bits, $s = 20$
\item Jacobi symbol $n$ 1024 bits, $s=20$
\item Hom. based on discrete logarithm $d$ 20 bits, $n$ 1024 bits, $s=1$
\item RSA $n$ 1024 bits, $s=1$
\end{itemize}
\end{slide}
%%%%%%%%%%%%%%%%%%%%%%%%%%%%%%%%%%%%%%%%%%%%%%%%%%%%%%
\begin{slide}
\slidetitle{Signature generation}
\\[1.0cm] 
\emph{Results}
 \begin{table}[htb] 
 \begin{tabular}{|l c|} 
 \hline 
Signature generation time&  {time in ms}\\ 
 \hline  
 Quartic residue symbol $(\beta=\pi\sigma)$ & 493.01\\ 
 Quartic residue symbol $(\beta=\pi)$ & 180.64\\
 Jacobi symbol (basic algorithm) & 25.22\\ 
 Jacobi Symbol (GMP) & 2.32\\  
 Discrete logarithm (Precomputed Table) & 9.66\\ 
 Discrete logarithm (BSGS) & 19.47\\ 
 Discrete logarithm (Pollard's rho) & 74.93\\ 
 RSA & 33.87\\
 \hline 
 \end{tabular}
 \end{table}
\end{slide}
%%%%%%%%%%%%%%%%%%%%%%%%%%%%%%%%%%%%%%%%%%%%%%%%%%%%%%
\begin{slide}
\slidetitle{Conclusion}\\[1cm]
\begin{itemize}
\item Reduced computation time for quartic residue symbol by more than half
\item $\mathbb{Z}[i]$ computations prevent quartic residue symbol from being competitive with Jacobi symbol
\item Compared implementations of additional homomorphisms by storage requirement and speed
\end{itemize}

\end{slide}
\end{document}
