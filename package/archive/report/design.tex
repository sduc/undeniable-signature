\chapter{Application Design}
\label{chap:design}
    
    The University Contest application is an Android application where students form teams to answer to quizzes from other universities and score points for their own university.
    The application uses MOVA to sign unfilled/filled quizzes.
    The architecture of the application will be described in this chapter. We will also discuss some implementation choices.

    \section{Architecture}
    \label{sec:archi}
    The application has the following architecture. It is composed of three participant: 
    the Android application on which the students mainly fill the quizzes, the server of the first university and the opposing university.
    
    \subsection{The Android Application}
    The Android application on the phone has three main activities~\cite{cite:android}:
    \begin{description}
        \item[ChallengeActivity] The activity allowing the user to fill the current quiz.
        \item[TeamScoreActivity] The activity used to get the score of the team and the result/correction of the quizzes.
        \item[UniversityScoreActivity] The activity allowing the user to obtain the score of all universities participating the contest.
    \end{description}
    The \texttt{ChallengeActivity} is the most useful activity. When starting, it first checks if the phone already knows the MOVA public keys/parameters of both servers. If not it 
    retrieves them from the respective server. Then it checks if there is an ongoing challenge already available on the phone. If not it tries to get a new one from the server of 
    the opposing university. Then the student can fill the quiz and has to send it before a deadline. When sending the quiz, it is first sent to the respective university of the students 
    which will sign it and send it back to the user. Then it automatically check the signature and sends it to the opposing university which will grade it.

    The \texttt{TeamScoreActivity} is an activity containing the list of all quizzes submitted by the team with the score and the responses. It also contains the cumulated score of the team.
    This part is not implemented yet.

    The \texttt{UniversityScoreActivity} is an activity containing the list of all universities participating the contest with their respective score.
    This part is not implemented yet.

    \subsection{The Servers}
    The two servers run the same custom protocol because of the symmetry of the application. To make simple, the client can
    \begin{itemize}
        \item Ask for the servers MOVA public key and MOVA parameters.
        \item Ask a new challenge (to the opposing university).
        \item Ask for his team score with the result of all quizzes his team submitted (to the opposing university).
        \item Ask for the score of all universities (to all servers, so by default it will ask his university server).
        \item Ask to sign his filled-in quiz (to his university server).
        \item Ask for the verification/denial of a message signed using MOVA.
        \item Send a filled-in quiz so that the server signs it (to his university server).
        \item Send a filled-in signed quiz so that the server grades it (to the opposing server).
    \end{itemize}
    The servers also communicate together to verify the signature of the filled quizzes they receive and to update the scores of the universities. 

    Let us discuss now how the server are implemented. All the code was made in Java~\cite{cite:java}.
    
    \subsubsection{Client Query Handler}
    This is a process that handles all queries from the client cited above. It is listening on port 12345 for new clients. 
    To start this process just run \texttt{UniversityContestServer.class}.
    
    \subsubsection{Database}
    It is a MySQL~\cite{cite:sql} database. The name of the database is \texttt{unicontest}. It is composed of three tables:
    \begin{description}
        \item[\texttt{team}] The team table contains the following informations. 
            The team ID, the score of the team, the university of the team,  and the ID of the current challenge for the team.
        \item[\texttt{challenge}] The challenge table contains the following entries: 
            the challenge ID, the associated team ID, the score that the team did for that challenge, a boolean value telling
            whether the filled-in challenge was received by the server, a boolean value telling whether the challenge was corrected and the name of the challenge.
        \item[\texttt{university}] The university table contains the following entries: the university name and the university cumulated score.
    \end{description}
    To initialize the database, \texttt{DataBaseInit.class} can be used. But before a \texttt{java} user has to be created with the appropriated password.
    The other processes access the database using class \texttt{DataBaseHandler}.
    
    \subsubsection{Server Manager}
    The server manager is used by the administrator of the server. It can be used to add a new team, a new university or a new challenge. It can also be used to manage the challenges,
    correct them and assign scores. For more comfort, a GUI is provided. The latter can be started by running \texttt{ServerManagerGUI.class}.
    
    \section{Implementation} 
    \label{sec:implementation}
    In this section we will discuss some implementation choices. First we will discuss how MOVA was implemented. Then we will
    discuss what choices where made for the commitment scheme and the pseudorandom generator (PRNG).

    \subsection{MOVA}
    Let us discuss how MOVA was implemented for the application. Given that MOVA is quiet theoretical, some implementation choices had to be made leading to some loss of generality 
    in the scheme.
    For the sake of simplicity and for efficiency reasons, the following choices were made.
    \begin{itemize}[label=,leftmargin=0.8cm,rightmargin=1cm]
        \item $\Xgroup = (\XdX{\Z}{n})^*$ for some $n=p_1 p_2$ which is the product of two large prime numbers. 
        \item $\Ygroup = \{-1,1\}$ leading to $d=p=2$.
        \item The homomorphisms are the Legendre symbols in $G$. Suppose without any 
            loss of generality that it is $\legendre{\cdot}{p_1}$
    \end{itemize}

    \noindent Therefore with these choices, the public key will be defined by:
    \[
        K_p^{S} = (n,\seedK,(\Ykey{1},...,\Ykey{\Lkey}))
    \]
    since $d$ is always 2 and Ygroup is always $\{-1,1\}$. The secret key will be defined as:
    \[
        K_s^{S} = p_1
    \]
    since $p_1$ totally defines the homomorphism.

    \subsection{Other Cryptographic Primitives}
    In MOVA confirmation/denial protocol, a commitment scheme is needed. In this project, the commitment scheme was implemented
    as in Example~\ref{example:commitment}. In this implementation a hash function is needed and SHA-256~\cite{nist:sha} 
    is used. 

    For the PRNG, we used the implementation in GNU cryto 2.1.0\cite{cite:gnu-crypto}.

    \section{The Code}
    Everything was coded in Java 6 and Android 2 for the application. The code is divided into three Eclipse~\cite{eclipse} projects.
    \begin{description}
        \item[Share] This project contains all shared code between the client and the server. It consists of the protocol,
            the MOVA scheme and all classes related to the application (such as the \texttt{Quiz} class).
        \item[UniversityContest] This project is the Android project containing the client part to be installed on the mobile
            devices
        \item[Server] This project contains the server part, which is the query handler and the server manager.
    \end{description}

    The code is commented in a Java Doc style so it will not be further discussed in this report.


