\documentclass[footer]{lasecposter}

\usepackage[latin1]{inputenc}
\usepackage[english]{babel}
\usepackage{amsmath, amssymb, amsfonts, verbatim}

% Prepare the title part

\title{Generic Homomorphic Undeniable Signature Scheme: Optimizations}
\author{Yvonne Anne Oswald}
\email{yvonneanne.oswald@epfl.ch}
\supervisor{Jean Monnerat}
\institution{Security and Cryptography Laboratory}
\date{February 2005}
\abstract{
A new undeniable signature scheme was proposed by LASEC. It allows to transform a private group homomorphism into a undeniable signature scheme. In 2004, a demonstrator for this scheme using the quartic residue symbol as a homomorphism has been implemented. The aim of this project was to optimize the existing implementation and to implement 3 additional homomorphisms (Jacobi symbol, discrete logarithm, RSA exponentiation) and compare them to each other.}

\begin{document}

\maketitle % create the title part
\begin{multicols}{2}

\subsection*{\sffamily Quartic residue and Jacobi symbol}
By optimizing and reimplementing we achieved to reduce the running time of the existing implementation of the basic algorithm for the quartic residue symbol by half. Furthermore, we tried out Damg\aa rd's algorithm as well as a mixed algorithm, which lead to another gain in speed, and compared them to implementations of the Jacobi symbol, which is the equivalent of the quartic residue symbol in $\mathbb{Z}$.
 
 
  \subsection*{\sffamily Other homomorphisms}
We implemented three versions of a homomorphism based on the discrete logarithm: the first using a hash table containing the precomputed values of the discrete logarithm, the second and the third implementing the Baby Step Giant Step (BSGS) and Pollard's Rho algorithm. We compared them to our implementation of the RSA exponentiation, the quartic residue symbol and the Jacobi symbol.  
\end{multicols}

\subsection*{\sffamily Results}
Our implementations are written in C and use the GNU Multiple Precision Arithmetic Library (GMP) to handle large integers. We conducted our timing measurements with  random numbers of the size a typical security level requires. All our results are average values.
 \begin{table}[htb] 
 \centering
 \sffamily 
 \begin{tabular}{|l c c|} 
 \hline 
Computation of one homomorphism &  {time in ms} &  {iterations}\\ 
 \hline  
 Quartic:~basic algorithm not optimized & 62.79 & 249.27\\
 Quartic:~basic algorithm optimized & 31.57 & 249.27\\
 Quartic:~Damg\aa rd's algorithm  &  50.63 & 766.12\\
 Quartic:~mixed algorithm & 24.65 & 511.92\\ 
 Jacobi:~~basic algorithm & 1.26 & 187.71\\
 Jacobi:~~binary algorithm (GMP) & 0.12 & \\
 Discrete logarithm: precomputed table & 9.66 & \\ 
 Discrete logarithm: BSGS& 19.47 & 388.36\\ 
 Discrete logarithm: Pollard's Rho& 74.93 & 1037.49\\ 
 RSA: & 33.87 & \\
 \hline 
 \end{tabular}
 \end{table}

\end{document}

